\documentclass[11pt,a4paper,twoside]{article}

% LaTeX-Umsetzung der "Richtlinien für Projekt- und Diplomarbeiten"
% der LFE Medieninformatik, LMU München. (Autor: Richard Atterer, 27.9.2006, 23.10.2007), Bug-Fixing Mark Kaczkowski (23.6.2008)

\usepackage[T1]{fontenc} % sonst geht \hyphenation nicht mit Umlauten
%\usepackage[latin1]{inputenc} % man kann schreiben äöüß statt "a"o"u"s
\usepackage[utf8]{inputenc} % wie oben, aber UTF-8 als Encoding statt ISO-8859-1 (latin1)
\usepackage[ngerman,english]{babel} % deutsche Trennregeln, "Inhaltsverzeichnis" etc.
%\usepackage{ngerman} % Alternative zum Babel-Paket oben
\usepackage{mathptmx} % Times-Roman-Schrift (auch für mathematische Formeln)

% Zum Setzen von URLs
\usepackage{color}
\definecolor{darkred}{rgb}{.25,0,0}
\definecolor{darkgreen}{rgb}{0,.2,0}
\definecolor{darkmagenta}{rgb}{.2,0,.2}
\definecolor{darkcyan}{rgb}{0,.15,.15}
\usepackage[plainpages=false,bookmarks=true,bookmarksopen=true,colorlinks=true,
  linkcolor=darkred,citecolor=darkgreen,filecolor=darkmagenta,
  menucolor=darkred,urlcolor=darkcyan]{hyperref}

% pdflatex: Bilder in den Formaten .jpeg, .png und .pdf
% latex: Bilder im .eps-Format
\usepackage{graphicx}

\usepackage{fancyhdr} % Positionierung der Seitenzahlen
\fancyhead[LE,RO,LO,RE]{}
\fancyfoot[CE,CO,RE,LO]{}
\fancyfoot[LE,RO]{\Roman{page}}
\renewcommand{\headrulewidth}{0pt}
\setlength{\headheight}{13.6pt} % behebt headheight Warning

% Korrektes Format für Nummerierung von Abbildungen (figure) und
% Tabellen (table): <Kapitelnummer>.<Abbildungsnummer>
\makeatletter
\@addtoreset{figure}{section}
\renewcommand{\thefigure}{\thesection.\arabic{figure}}
\@addtoreset{table}{section}
\renewcommand{\thetable}{\thesection.\arabic{table}}
\makeatother

\sloppy % Damit LaTeX nicht so viel über "overfull hbox" u.ä. meckert

% Ränder
\addtolength{\topmargin}{-16mm}
\setlength{\oddsidemargin}{25mm}
\setlength{\evensidemargin}{35mm}
\addtolength{\oddsidemargin}{-1in}
\addtolength{\evensidemargin}{-1in}
\setlength{\textwidth}{15cm}
\addtolength{\textheight}{34mm}
%______________________________________________________________________

\begin{document}

\pagestyle{empty} % Vorerst keine Seitenzahlen
\pagenumbering{alph} % Unsichtbare alphabetische Nummerierung

\begin{center}
\textsc{Ludwig-Maximilians-Universität München}\\
Department ``Institut für Informatik''\\
Lehr- und Forschungseinheit Medieninformatik\\
Prof.\ Dr.\ Heinrich Hußmann

\vspace{5cm}
{\large\textbf{Bachelor's Thesis}}\vspace{.5cm}

{\LARGE Using Keyword Extraction for Image Retrieval in News Articles}\vspace{.3cm}

{\Large Comparison of Different Approaches}\vspace{1cm}

{\large Martin Schön}\\\href{mailto:martingeorg.schoen@gmail.com}{martingeorg.schoen@gmail.com}

\end{center}
\vfill

\begin{tabular}{ll}
Bearbeitungszeitraum: & 1. 6. 2018 bis 19. 10. 2018\\
%Externer Betreuer: & Manfred Manager\\
Verantw. Hochschullehrer: & Prof. Dr. Neil Thurman
\end{tabular}
%______________________________________________________________________

\clearpage
\selectlanguage{english}
\section*{Abstract}

Short abstract of the work, maximum of 250 words.

\selectlanguage{ngerman}
\clearpage
\section*{Aufgabenstellung}

Kopie der Original-Aufgabenstellung

\vfill % Sorgt dafür, dass das Folgende an das Seitenende rutscht

\noindent Ich erkläre hiermit, dass ich die vorliegende Arbeit
selbstständig angefertigt, alle Zitate als solche kenntlich gemacht
sowie alle benutzten Quellen und Hilfsmittel angegeben habe.

\bigskip\noindent Berlin, \today

\vspace{4ex}\noindent\makebox[7cm]{\dotfill}

%______________________________________________________________________

\selectlanguage{english}
\cleardoublepage
\pagestyle{fancy}
\pagenumbering{roman} % Römische Seitenzahlen
\setcounter{page}{1}

% Inhaltsverzeichnis erzeugen
\tableofcontents

%Abbildungsverzeichnis erzeugen - normalerweise nicht nötig
%\cleardoublepage
%\listoffigures
%______________________________________________________________________

\cleardoublepage

% Arabische Seitenzahlen
\pagenumbering{arabic}
\setcounter{page}{1}
% Geändertes Format für Seitenränder, arabische Seitenzahlen
\fancyhead[LE,RO]{\rightmark}
\fancyhead[LO,RE]{\leftmark}
\fancyfoot[LE,RO]{\thepage}

\section{Introduction}

Oh Mann, noch \emph{so} viele Seiten zu füllen...\\
...und wieso muss bei diesem Format so viel auf eine Seite passen!?
%______________________________________________________________________

\cleardoublepage

\section{Theoretical Background}

\subsection{Images and News Perception}

\subsection{Related Image Retrieval Approaches}

%______________________________________________________________________

\cleardoublepage

\section{Proposed System}

\subsection{System Overview}

\subsection{Corpus}

\subsection{Preprocessing an Article}
\subsubsection{Features of a Term}

\subsection{Training Neural Networks}
\subsubsection{Preprocessing an Image for Training}
\subsubsection{Generating Training Data}
\subsubsection{Training the Networks}

\subsection{Term Classification}
\subsubsection{Statistics Based Approach}
\subsubsection{Machine Learning Based Approach}

\subsection{Image Query}

%______________________________________________________________________

\cleardoublepage

\section{Evaluation}

\subsection{Factual Correctness as a Measure for System Performance}

\subsection{Evaluation Design}

\subsection{Evaluation Results}

%______________________________________________________________________

\cleardoublepage

\section{Limitations}

\subsection{System Limitations}

\subsection{Research Design Limitations}

%______________________________________________________________________

\cleardoublepage

\section{Conclusion}

%______________________________________________________________________

% Der Befehl \cleardoublepage erscheint nur vor \section, nicht vor
% den "kleineren" Gliederungsbefehlen wie \subsection!
\cleardoublepage % Neue rechte Seite anfangen
\section{Example Section}

\begin{figure}%[btph]
  %% Datei ``beispielbild.eps'' oder ``beispielbild.png'', zentriert
  %\begin{center}\includegraphics{beispielbild}\end{center}

  %% Datei auf 8cm Breite verkleinert/vergrößert
  %\includegraphics[width=8cm]{beispielbild}
  %% Datei auf ganze Breite des Texts vergrößert
  %\includegraphics[width=\columnwidth]{beispielbild}
  %% Datei auf 60% der Textbreite verkleinert/vergrößert
  %\includegraphics[width=.6\columnwidth]{beispielbild}
  %% Weitere Optionen (Ausschnitt, drehen etc.) in der Doku zum graphicx-Paket

  \begin{center}\LARGE [BILD]\end{center}
  \caption{Bildunterschrift}
  \label{fig:beispielbild}
\end{figure}


Siehe Abbildung \ref{fig:beispielbild} oder einschlägige Literatur, z.B.
\cite[Seite 6]{Brill1992ATagger} oder \cite{Porter1980AnStripping}.

\bigskip % Größerer Abstand zum vorherigen Absatz
\textbf{Hinweis:} Die Verweise im generierten PDF (HTTP-Links, Verweise auf Kapitel oder Bilder) sind leicht eingefärbt. Wer das nicht will, z.B. weil es die Druckkosten erhöht, kann am Anfang des Dokuments \texttt{linkcolor} usw. auf ``black'' setzen.


\subsection{Medien}

\begin{figure}
  \begin{center}\LARGE [BILD]\end{center}
  \caption{Noch ein Bild}
  \label{fig:beispielbild2}
\end{figure}

\begin{itemize}
  \item Gesellschaftliche Medien
  \item Technische Medien
\end{itemize}


\subsection{Informatik}


\subsection{Medieninformatik}

\begin{description}
  \item[Medienwirkung:] Ein Spezialfach der Kommunikationswissenschaft. Für das erfolgreiche Studium des Anwendungsfachs Mediengestaltung ist eine künstlerische Begabung erforderlich.
  \item[Medienwirtschaft:] Ein Spezialfach der Betriebswirtschaftslehre
  \item[Mediengestaltung:] Ein Spezialfach der Kunstwissenschaft
\end{description}

\subsubsection{Was Sie schon immer wissen wollten, aber nie zu fragen
  wagten}

\paragraph{Überschrift}
Diese Überschrift erscheint fettgedruckt am Anfang des Absatzes.

\subsubsection{Was Sie nicht wissen wollten}

Text text textextext\footnote{Oder so ähnlich}.

%\_____________________________________________________________________

\cleardoublepage
\fancyhead[LE,RO,LO,RE]{} % Keine Kopfzeile mehr oben auf jeder Seite
\section*{Contents of the Attached CD}
%______________________________________________________________________

\cleardoublepage
%\begin{thebibliography}{99}

%\bibitem{Ivory01}

%  M.\ Y.\ Ivory, M.\ Hearts:
%  \href{http://www.ischool.washington.edu/myivory/thesis/thesis.pdf}{%
%    An Empirical Foundation for Automated Web Interface Evaluation}.
%  Ph.D. thesis, University of California at Berkeley, 2001


%\cleardoublepage
%\hspace{-\leftmargin}{\Large\bfseries Web-Referenzen} % Wüster Hack %-|

%\bibitem{NielsenAlertbox}

%  J.\ Nielsen: Alertbox: Current Issues in Web Usability
%  \url{http://useit.com/alertbox/}, accessed April~24, 2005.

%\end{thebibliography}

\bibliographystyle{plain}
\bibliography{mendeley_v2}

\end{document}
